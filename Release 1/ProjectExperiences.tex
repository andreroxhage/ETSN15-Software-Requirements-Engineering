\documentclass[a4paper]{article}
\usepackage[T1]{fontenc}
\usepackage[english]{babel}
\usepackage[utf8]{inputenc}
\usepackage[margin=3cm]{geometry}
% \usepackage{amsmath}
% \usepackage{amssymb}
\usepackage{parskip}
% \usepackage{csquotes}
% \usepackage[hyperref,backend=biber,style=ieee,isbn=true,urldate=iso8601]{biblatex}
\usepackage[colorlinks=true, allcolors=blue]{hyperref}
% \usepackage{listings}
\usepackage{graphicx}

%% Hyperref options
\makeatletter
\hypersetup{
    linktoc=all,
    unicode=true,
    pdfcreator=,
    pdftitle={\@title},
    pdfauthor={\@author},
    pdfdisplaydoctitle=true,
}

\title{Project Experiences}
\author{Jonathan Ahlström \and Ossian Gewert \and Jacob Jönsson \and Simon Persson \and André Roxhage \and Felix Sundholm}

\begin{document}
\maketitle

\begin{center}
    ETSN15 Requirements Engineering

Group Gamma - EasyTrip

% Jonathan Ahlström
% Ossian Gewert
% Jacob Jönsson
% Simon Persson
% André Roxhage
% Felix Sundholm
\end{center}

\tableofcontents

\newpage

% (a) Description of your requirements engineering work, including experiences and reflections in relation to learning objectives.
% (b) Description of the chosen methods/techniques for elicitation, specification, validation, and prioritization.
% (c) Motivation for why you chose the used methods/techniques.
% (d) Reflection on the usage of these methods/techniques in terms of what
% was successful and what was challenging. Example questions for reflection: What have you learned in relation to the learning objectives in
% this course program? What would you have done differently based on
% what you know now? What have you learned in relation to the learning
% objectives?
% (e) Reflection on the interaction among internal and (potential) external
% stakeholders through the different steps of the Requirements Engineering process.
% (f) A personal statement by each team member that briefly explains each
% individual’s contributions to the project results (what) and the total
% number of hours of project work per person (how much).
% (g) The Project Experiences should not include course evaluation issues,
% but focus on your own work and learning outcome.

% Elicitation Reflection:
% Reflection on the elicitions performed, what went well, what could have been done better
\section{Elicitation Reflection}

% (a) Description of your requirements engineering work, including experiences and reflections in relation to learning objectives.
% (b) Description of the chosen methods/techniques for elicitation, specification, validation, and prioritization.
% (c) Motivation for why you chose the used methods/techniques.
% (d) Reflection on the usage of these methods/techniques in terms of what
% was successful and what was challenging. Example questions for reflection: What have you learned in relation to the learning objectives in
% this course program? What would you have done differently based on
% what you know now? What have you learned in relation to the learning
% objectives?
% (e) Reflection on the interaction among internal and (potential) external
% stakeholders through the different steps of the Requirements Engineering process.
% (f) A personal statement by each team member that briefly explains each
% individual’s contributions to the project results (what) and the total
% number of hours of project work per person (how much).
% (g) The Project Experiences should not include course evaluation issues,
% but focus on your own work and learning outcome.

\subsection{Context diagram}

Firstly we created a context diagram from our understanding which we later discovered lacked important entities. This work broadened our knowledge of context diagram and view on work in requirements engineering. We discovered that we didn’t know exactly how the product could earn money. After some discussion we came to the conclusion that there will be two separate sources; ads and referrals to airline companies. We also had a healthy discussion about which actors were in direct contact with the product. We decided that travel agencies may want to use our product but that it would not be our main target user. It was also concluded that we should offer support and that they would be a stakeholder as well. 

From our discussions with our supervisor we learned that there are more stakeholders we need to take into consideration compared to what we previously had listed. This certainly helped the product develop even further and enhance its project description to formulate a clearer goal view. 

We decided to create a context diagram because of two main reasons, firstly it helps us create an architecture for the project that we can later use to describe our different features and relations. Secondly the technique helps narrow down what actors are involved with the product and who the stakeholders are.

One challenge we discovered was that the team was not unified on what entities or actors are supposed to be placed in a certain state. Such as that our ad provider is supposed to be inside or outside of the inner domain.

What we recommend for others to do is firstly shorten down your thought process and focus on one aspect at a time. For example, firstly look at direct actors inside the inner domain, and then move on one step at a time to minimize risk of missing components or misplacing entities.

Overall we thought that the method used was very useful in an early stage for elicitation, requirement generation and to get everyone involved on the same page.




% Specification Reflection:
% Reflection on the specification process, what went well, what could have been done better
\section{Specification Reflection}
\subsection{Data Requirements}

For the data requirements three methods were chosen. These methods were data model, data dictionary and virtual window. For this project one experience was that
making data requirements are more or less difficult depending on what part of the project is developed. The data requirements used were thus targeting the user
account system in the product as well as the search system for the flights. These are the crucial parts that are the most important to include in early releases.
Other areas such as the ad providers or how the data from airlines should be handled are harder to construct and a reason for this is a lack of knowledge in
how this data would be represented from thses sources, this would be improved over the release iterations in the project over time.

Another point that was explored was in which level the E/R-model was placed. Initially it was placed in the design level but since data requiremtns usually are similar
across the different levels it was chosen to be moved up to domain level instead in order to not interfere with the design level implementations.

% Initial Challenges Reflection:
% Reflection on the initial challenges faced, what went well, what could have been done better
\section{Initial Challenges Reflection}
\subsection{Restructured requirements specification}
During the development of our requirements specification, we initially structured the document by categorizing requirements into distinct sections such as functional, quality, and data requirements. While this approach provided a clear separation of requirement types as we learned about them in the course, we encountered difficulties in effectively contextualizing them within specific features and functionalities of the system.

To address this, we restructured our requirements specification into sections based on key system areas, such as the heatmap, security, and search functionalities. This new structure allows all relevant requirements, regardless of type, to be included within the appropriate section, making it easier to see how different levels of requirements interact within a particular feature. By organizing the document this way, we have improved traceability, making it more intuitive to understand how individual requirements contribute to the overall system design and objectives. Additionally, this restructuring has resulted in fewer cross-references, simplifying navigation and reducing the complexity of managing related requirements.

Also, we updated our requirement naming convention from numbered identifiers to camelCase IDs. This change enhances clarity and consistency across the document while improving readability and maintainability. The camelCase format ensures that requirement identifiers are more descriptive and making cross-referencing simpler when working with different system components.

These refinements have significantly enhanced the structure and readability of our requirements specification, ensuring that it better supports both development and validation processes.

\subsection{Release planning and prioritization}

We used three different methods for prioritazation, benefits cost analysis, priority classification and prioritazation though constraint-based release Planning in RecT. These methods were combined in order to create a interative release plan that maximised value while costing as little as possible. To use only one of these would make the release planning more susceptible to inconsistencys. This is why it was important for us to make the techniques independent from each other.

One of the main difficulties was balancing the needs of different stakeholders while also managing dependencies between features. Many subjective estimations can be harmful for the result since the prioritaztion is not made in a systematic way. Manually handling these dependencies proved too complex which is why  we decided to use the tool RecT to automate the process. The score of the relases and features that RecT gave us were somewhat ambiguous since fictitious estimates were used in benefit and cost. This made the result an abstract number which actually was negative sometimes. Even though this part of the result was not desirable, the tool provided a useful relative comparison between features. The release plan RecT generated, with realeses R4, R5 and R6, seemed realistic given our features and constraints. Looking back, we would have benefited from using RecT earlier in the project. This would have given us more time to refine our dependencies and prioritize features more effectively. 

Benefits cost analysis was chosen because it provides a quantifiable way to compare features based on benefit and cost. This was a good way further analysize our stakeholders, noticing how their needs can differ and sometimes contradict each other. While travels value usability, the travel agency might prioritize safely for example. A scale of only three options, “Must-Have”, “Important” \& “Nice to have”, was chosen to was chosen to avoid unnesesary complexity and ensure a clear prioritization.

Constraint based priorizitaion seemed like a logical way to make the solution space smaller which would lead us closer to a more realistic release plan. After manually connecting different types of dependencies, “Precedence”, ”Coupling” \& ”Excludes”, to features, we discussed how the types of dependencies varied from each other and when each should be applied. By discussing this our understanding out our insight on how the features relate to each other got more clear. We also discussed how the complexity would be much larger in a real world scenario as every dependency would be more important to specify. 


\section{Validation Reflection}
\subsection{Validation List}
What did we learn from making the validation list?
Writing a report when validating someone elses work?
What feedback did we recieve?


\subsection{Prototyping for Validation}
Prototyping can play a crucial role in validating our concept for EasyTrip, ensuring that both usability and functionality aligned with user expectations. Since we had already validated the idea during the elicitation phase, the prototyping stage focused on creating a representation of the needs and requirements identified.
\subsubsection{Prototyping}
First, we developed rough sketches to outline the core functionalities and user flows. These sketches helped visualize the interactions between users and system components early in the process. Our team discussions refined these ideas before transitioning into digital prototyping.

Next, we created wireframes using Figma to map out key user journeys, including searching for flights, filtering results, and accessing booking options. To simulate interactions more realistically, we wanted to build an interactive prototype incorporating dynamic elements. However, due to time constraints, it resulted in a static prototype without interactions.

Despite not being able to test features, the static prototype allowed us to evaluate the overall layout, navigation, and content structure internally. Feedback gathered from team members provided insights into information clarity and flow, helping us refine key design elements, as well as made sure the team where on the same page and has common expectations of what we were building. The prototype also enabled us to align our vision with stakeholder expectations before committing to full-scale development.

\subsubsection{Future Directions and Improvements}

To enhance future validation efforts, we propose incorporating interactive elements in the prototypes. One approach is to use dummy data to create a high-fidelity prototype with basic search and filter interactions, allowing for more realistic user testing.

Expanding usability testing to include a diverse group of stakeholders, such as frequent travelers, travel agencies, and airline representatives, would provide a more comprehensive understanding of user needs and potential system constraints.

Additionally, A/B testing different UI layouts, such as list-based search results versus heatmap visualizations, would further refine the interface based on user preferences.

By adopting these improvements, we can bridge the gap between conceptual validation and real-world usability testing, ensuring a more robust and user-centered development process before full-scale implementation.

\section*{Individual Contributions}

\begin{tabular}{|l|p{10cm}|}
    \hline
    \textbf{Group Member} & \textbf{Contributions} \\
    \hline
    Jonathan Ahlström & Prototype. Requirements. Release plan. Requirements structure. Release plan and prioritization reflections \\
    \hline
    Ossian Gewert & ? \\
    \hline
    Jacob Jönsson & Data Requirements. Data Requirements Reflections. A few Requirements. \\
    \hline
    Simon Persson & Requirements. Minor Release plan. Requirements structure. Quality grid. Quality requirements reflection, Presentation \\
    \hline
    André Roxhage & Elicitation interview, stakeholder analysis, prototyping, release plan \& prioritization.  \\
    \hline
    Felix Sundholm & Virtual window reflection, context diagram reflection, presentation slides \& presentation \\
    \hline
\end{tabular}
\end{document}
