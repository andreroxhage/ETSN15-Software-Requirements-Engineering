\documentclass[a4paper]{article}
\usepackage[T1]{fontenc}
\usepackage[english]{babel}
\usepackage[utf8]{inputenc}
\usepackage[margin=3cm]{geometry}
% \usepackage{amsmath}
% \usepackage{amssymb}
\usepackage{parskip}
% \usepackage{csquotes}
% \usepackage[hyperref,backend=biber,style=ieee,isbn=true,urldate=iso8601]{biblatex}
\usepackage[colorlinks=true, allcolors=blue]{hyperref}
% \usepackage{listings}
\usepackage{graphicx}

%% Hyperref options
\makeatletter
\hypersetup{
    linktoc=all,
    unicode=true,
    pdfcreator=,
    pdftitle={\@title},
    pdfauthor={\@author},
    pdfdisplaydoctitle=true,
}

\title{Project Experiences}
\author{Jonathan Ahlström \and Ossian Gewert \and Jacob Jönsson \and Simon Persson \and André Roxhage \and Felix Sundholm}

\begin{document}
\maketitle

\begin{center}
    ETSN15 Requirements Engineering

Group Gamma

Jonathan Ahlström
Ossian Gewert
Jacob Jönsson
Simon Persson
André Roxhage
Felix Sundholm
\end{center}

\tableofcontents

% (a) Description of your requirements engineering work, including experiences and reflections in relation to learning objectives.
% (b) Description of the chosen methods/techniques for elicitation, specification, validation, and prioritization.
% (c) Motivation for why you chose the used methods/techniques.
% (d) Reflection on the usage of these methods/techniques in terms of what
% was successful and what was challenging. Example questions for reflection: What have you learned in relation to the learning objectives in
% this course program? What would you have done differently based on
% what you know now? What have you learned in relation to the learning
% objectives?
% (e) Reflection on the interaction among internal and (potential) external
% stakeholders through the different steps of the Requirements Engineering process.
% (f) A personal statement by each team member that briefly explains each
% individual’s contributions to the project results (what) and the total
% number of hours of project work per person (how much).
% (g) The Project Experiences should not include course evaluation issues,
% but focus on your own work and learning outcome.

% Elicitation Reflection:
% Reflection on the elicitions performed, what went well, what could have been done better
elicitationReflection.tex


% Specification Reflection:
% Reflection on the specification process, what went well, what could have been done better
specificationReflection.tex

% Initial Challenges Reflection:
% Reflection on the initial challenges faced, what went well, what could have been done better
initialChallengesReflection.tex

\end{document}