\subsection{Validation List}
What did we learn from making the validation list?
Writing a report when validating someone elses work?
What feedback did we recieve?


\subsection{Prototyping for Validation}
Prototyping can play a crucial role in validating our concept for EasyTrip, ensuring that both usability and functionality aligned with user expectations. Since we had already validated the idea during the elicitation phase, the prototyping stage focused on creating a representation of the needs and requirements identified.
\subsubsection{Prototyping}
First, we developed rough sketches to outline the core functionalities and user flows. These sketches helped visualize the interactions between users and system components early in the process. Our team discussions refined these ideas before transitioning into digital prototyping.

Next, we created wireframes using Figma to map out key user journeys, including searching for flights, filtering results, and accessing booking options. To simulate interactions more realistically, we wanted to build an interactive prototype incorporating dynamic elements. However, due to time constraints, it resulted in a static prototype without interactions.

Despite not being able to test features, the static prototype allowed us to evaluate the overall layout, navigation, and content structure internally. Feedback gathered from team members provided insights into information clarity and flow, helping us refine key design elements, as well as made sure the team where on the same page and has common expectations of what we were building. The prototype also enabled us to align our vision with stakeholder expectations before committing to full-scale development.

\subsubsection{Future Directions and Improvements}

To enhance future validation efforts, we propose incorporating interactive elements in the prototypes. One approach is to use dummy data to create a high-fidelity prototype with basic search and filter interactions, allowing for more realistic user testing.

Expanding usability testing to include a diverse group of stakeholders, such as frequent travelers, travel agencies, and airline representatives, would provide a more comprehensive understanding of user needs and potential system constraints.

Additionally, A/B testing different UI layouts, such as list-based search results versus heatmap visualizations, would further refine the interface based on user preferences.

By adopting these improvements, we can bridge the gap between conceptual validation and real-world usability testing, ensuring a more robust and user-centered development process before full-scale implementation.