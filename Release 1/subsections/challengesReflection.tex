\subsection{Restructured requirements specification}
During the development of our requirements specification, we initially structured the document by categorizing requirements into distinct sections such as functional, quality, and data requirements. While this approach provided a clear separation of requirement types as we learned about them in the course, we encountered difficulties in effectively contextualizing them within specific features and functionalities of the system.

To address this, we restructured our requirements specification into sections based on key system areas, such as the heatmap, security, and search functionalities. This new structure allows all relevant requirements, regardless of type, to be included within the appropriate section, making it easier to see how different levels of requirements interact within a particular feature. By organizing the document this way, we have improved traceability, making it more intuitive to understand how individual requirements contribute to the overall system design and objectives. Additionally, this restructuring has resulted in fewer cross-references, simplifying navigation and reducing the complexity of managing related requirements.

Also, we updated our requirement naming convention from numbered identifiers to camelCase IDs. This change enhances clarity and consistency across the document while improving readability and maintainability. The camelCase format ensures that requirement identifiers are more descriptive and making cross-referencing simpler when working with different system components.

These refinements have significantly enhanced the structure and readability of our requirements specification, ensuring that it better supports both development and validation processes.

\subsection{Release planning and prioritization}
When deciding how to conduct a release plan, some discussion came up. One of the main difficulties was balancing the needs of different stakeholders while also managing dependencies between features. We discussed how many subjective estimations could be harmful for the result since they are not done in a systematic way. Manually handling these dependencies proved too complex, so we decided to use the tool RecT to automate the process. After manually defining the dependencies “Precedence”, ”Coupling” & ”Excludes” we got our features prioritized in a release plan. A lot of time was put into discussing how the types of dependencies varied from each other and when each should be applied. By discussing this our understanding out our features got more clear, as we now had to think about the them in greater detail. We discussed how the complexity would be much larger in a real world scenario as every dependency would be more important to specify. 

The score of the relases and features that RecT gave us were somewhat ambiguous since fictitious estimates were used in benefit and cost. This made the result an abstract number which actually was negative sometimes. Even though this part of the result was not desirable, the tool provided a useful relative comparison of benefit versus cost. The release plan RecT generated, with realeses R4, R5 and R6, seemed realistic given our features and constraints. Looking back, we would have benefited from using RecT earlier in the project. This would have given us more time to refine our dependencies and prioritize features more effectively. 

Challenges met:
\begin{itemize}
    \item DateRangeMap precedes DateRangeSearch (the map must support flexible dates before search can use them)
    \item ListFlightsFromMap requires ShowFlights (list view depends on retrieving flight details)
    \item SearchCheapFlights precedes ListCheapFlights (flights must be found before they can be listed)
    \item ListCheapFlights requires SearchCheapFlights (the listing function depends on the search results)
    \item SearchCheapDestinations precedes ListCheapFlights (finding cheap destinations must occur before listing them)
    \item DateRangeSearch requires DateRangeMap (searching by date range depends on the map supporting this feature first)
    \item ShowFlights requires SearchCheapFlights (users must search for flights before being able to book them)
    \item SaveFavoriteCity requires AccountCreation (only registered users can save favorite cities)
    \item AccountCreation precedes EmailConfirmation (accounts must be created before confirming email)
    \item EmailConfirmation requires CreateAccountEmail (email confirmation depends on users creating accounts first)
    \item CreateAccountEmail precedes EmailConfirmation (email account creation must occur before confirmation)
    \item EmailChange requires AccountCreation (users must have an account before changing their email)
    \item ChangePassword requires AccountCreation (users must have an account before changing their password)
    \item PasswordReset requires AccountCreation (users must have an account before requesting a password reset)
    \item MultipleTicketPrices requires ShowFlights (ticket prices must be displayed for selected flights)
    \item MultiFlightTrips requires MultipleTicketPrices (multi-leg trips depend on viewing multiple ticket prices)
\end{itemize}
