\subsection{Restructured requirements specification}
During the development of our requirements specification, we initially structured the document by categorizing requirements into distinct sections such as functional, quality, and data requirements. While this approach provided a clear separation of requirement types as we learned about them in the course, we encountered difficulties in effectively contextualizing them within specific features and functionalities of the system.

To address this, we restructured our requirements specification into sections based on key system areas, such as the heatmap, security, and search functionalities. This new structure allows all relevant requirements, regardless of type, to be included within the appropriate section, making it easier to see how different levels of requirements interact within a particular feature. By organizing the document this way, we have improved traceability, making it more intuitive to understand how individual requirements contribute to the overall system design and objectives. Additionally, this restructuring has resulted in fewer cross-references, simplifying navigation and reducing the complexity of managing related requirements.

Also, we updated our requirement naming convention from numbered identifiers to camelCase IDs. This change enhances clarity and consistency across the document while improving readability and maintainability. The camelCase format ensures that requirement identifiers are more descriptive and making cross-referencing simpler when working with different system components.

These refinements have significantly enhanced the structure and readability of our requirements specification, ensuring that it better supports both development and validation processes.