
% (a) Description of your requirements engineering work, including experiences and reflections in relation to learning objectives.
% (b) Description of the chosen methods/techniques for elicitation, specification, validation, and prioritization.
% (c) Motivation for why you chose the used methods/techniques.
% (d) Reflection on the usage of these methods/techniques in terms of what
% was successful and what was challenging. Example questions for reflection: What have you learned in relation to the learning objectives in
% this course program? What would you have done differently based on
% what you know now? 
% (e) Reflection on the interaction among internal and (potential) external
% stakeholders through the different steps of the Requirements Engineering process.
% (f) A personal statement by each team member that briefly explains each
% individual’s contributions to the project results (what) and the total
% number of hours of project work per person (how much).
% (g) The Project Experiences should not include course evaluation issues,
% but focus on your own work and learning outcome.

% Felix sammanfattar reflektion:
% (a) Beskriv arbetet, erfarenheter, reflektioner       (i relation till learning objectives, se kursprog.)
% (b) Beskriv metoden
% (c) Beskriv varför vi valde metoden
% (d) Reflektera över vad som gick bra, vad som var svårt
%   - Vad du lärt dig (learning objectives)
%   - Vad hade du gjort annorlunda? 
% (e) Reflektera över interaktioner mellan stakeholders     (inre och eventuellt yttre stakeholders)

%ADD SUBSUBSECTIONS
\subsection{Context diagram}

\subsubsection{Experiences}
The context diagram was done by the whole group together during an exercise. We learned about this quick and useful elicition technique.
\subsubsection{Description of the method}
A context diagram was made to elicitate which stakeholders there are and what interactions can exist between actors and the product. The result is a drawing showing the product as a black box, actors and users that interact with eachother and the product. The inner and outer domain is defined by drawing a circle around actors that interact directly with the product.
\subsubsection{Why this method?}
The method helps to understand what actors and interactions exist around the product. Conflicts of interests between different stakeholders can also be found through this method. It also produces a diagram that can be presented to others as an overview over interactions and potential stakeholders in an understandable way. The diagram can be a useful reference for future discussion and can also be used as a requirement. The method takes relatively short time and can be done in an early state to get everyone involved on the same page. It gets people talking and misunderstandings can be sorted out early.
\subsubsection{How it was}
The technique showed us that we knew very little about what actors could be involved and how they would interact. It also showed us that we had different ideas about how things would work. We discovered for example that we didn’t know exactly how the product could generate revenue. After some discussion, we concluded that there will be two separate sources; ads and referrals to airline companies. 

We also had a healthy discussion about which actors were in direct contact with the product. We decided that travel agencies may want to use our product but that it would not be our main target user. It was also concluded that we should offer support and that they would be a stakeholder as well. 

One challenge we discovered was that the team was not unified on what entities or actors are supposed to be placed in a certain state. Such as if our ad platform was supposed to be inside or outside of the inner domain or if we want to support travel agencies as well as travellers to use our product or only travellers.

Overall we thought that the method used was very useful in an early stage for elicitation.
\subsubsection{What we would do differently}
What we recommend for others to do is firstly shorten down your thought process and focus on one aspect at a time. For example, firstly look at direct actors inside the inner domain, and then move on one step at a time to minimize risk of missing components or misplacing entities.

\subsection{Virtual window}
\subsubsection{Experiences}
The virtual window method was done by the group together during an exercise.
\subsubsection{Description of the method}
The data requirements for an object (in a programming sense) are described by making a virtual window of what a form might look like that could be filled in to create any instance of the object. Example data are to be used to make it easier to understand.
\subsubsection{Why this method?}
The method is quite quick and gives an overview that is easy to understand for all readers. The example data can be much easier to understand than a long description such as in data dictionaries.
\subsubsection{How it was}
The method was found to be quick and intuitive to use. It helped us elicitate some requirements for a user such as if they would need to give consent to get emails about news from us.
\subsubsection{What we would do differently}
If anything we probably should have used this method more. For example we didn't delve too deeply into exactly how flight data would be handled since we would get it from the airline companies and we didn't know what that would look like. Still we could have used this method to elicitate some of the data requirements to get a better overview of what might come. The virtual window created this way could later be updated as we learned new things. This would work well since the method describes what data is needed but not how it should be implemented.

\subsection{Elicitation}
\subsubsection{Stakeholder Analysis}
The first and main elicitation made was the stakeholder analysis where we identified the stakeholders and their interests. This was done by brainstorming and discussing with the team during exercises in school. We later moved on letter two of the group members to write and deeply analyse the stakeholders.

\subsubsection{Traveller Interviews}
The second elicitation was of the stakeholder traveller where we interviewed two persons who travels frequently. This was done by two group members and the results were later summarised. We felt that it was a good way of getting a deeper understanding of the user and their needs. 

\subsubsection{Email Questions - Airlines and Travel Agencies}
The third elicitation was the sending of questions by email to the stakeholders airlines and travel agencies. This was done by two group members and the results were later summarised. Happiliy we got a response from one of the airlines, easyJet, which gave us some details but not as much as we hoped for. Although, we felt that the answers were still useful and validated reflections from the stakeholder analysis that was alreadty done. We did ask for a meeting but they declined.
