
% (a) Description of your requirements engineering work, including experiences and reflections in relation to learning objectives.
% (b) Description of the chosen methods/techniques for elicitation, specification, validation, and prioritization.
% (c) Motivation for why you chose the used methods/techniques.
% (d) Reflection on the usage of these methods/techniques in terms of what
% was successful and what was challenging. Example questions for reflection: What have you learned in relation to the learning objectives in
% this course program? What would you have done differently based on
% what you know now? What have you learned in relation to the learning
% objectives?
% (e) Reflection on the interaction among internal and (potential) external
% stakeholders through the different steps of the Requirements Engineering process.
% (f) A personal statement by each team member that briefly explains each
% individual’s contributions to the project results (what) and the total
% number of hours of project work per person (how much).
% (g) The Project Experiences should not include course evaluation issues,
% but focus on your own work and learning outcome.

\subsection{Context diagram}

Firstly we created a context diagram from our understanding which we later discovered lacked important entities. This work broadened our knowledge of context diagram and view on work in requirements engineering. We discovered that we didn’t know exactly how the product could earn money. After some discussion we came to the conclusion that there will be two separate sources; ads and referrals to airline companies. We also had a healthy discussion about which actors were in direct contact with the product. We decided that travel agencies may want to use our product but that it would not be our main target user. It was also concluded that we should offer support and that they would be a stakeholder as well. 

From our discussions with our supervisor we learned that there are more stakeholders we need to take into consideration compared to what we previously had listed. This certainly helped the product develop even further and enhance its project description to formulate a clearer goal view. 

We decided to create a context diagram because of two main reasons, firstly it helps us create an architecture for the project that we can later use to describe our different features and relations. Secondly the technique helps narrow down what actors are involved with the product and who the stakeholders are.

One challenge we discovered was that the team was not unified on what entities or actors are supposed to be placed in a certain state. Such as that our ad provider is supposed to be inside or outside of the inner domain.

What we recommend for others to do is firstly shorten down your thought process and focus on one aspect at a time. For example, firstly look at direct actors inside the inner domain, and then move on one step at a time to minimize risk of missing components or misplacing entities.

Overall we thought that the method used was very useful in an early stage for elicitation, requirement generation and to get everyone involved on the same page.

\subsection{Elicitation}
\subsubsection{Stakeholder Analysis}
The first and main elicitation made was the stakeholder analysis where we identified the stakeholders and their interests. This was done by brainstorming and discussing with the team during exercises in school. We later moved on letter two of the group members to write and deeply analyse the stakeholders.

\subsubsection{Traveller Interviews}
The second elicitation was of the stakeholder traveller where we interviewed two persons who travels frequently. This was done by two group members and the results were later summarised. We felt that it was a good way of getting a deeper understanding of the user and their needs. 

\subsubsection{Email Questions - Airlines and Travel Agencies}
The third elicitation was the sending of questions by email to the stakeholders airlines and travel agencies. This was done by two group members and the results were later summarised. Happiliy we got a response from one of the airlines, easyJet, which gave us some details but not as much as we hoped for. Although, we felt that the answers were still useful and validated reflections from the stakeholder analysis that was alreadty done. We did ask for a meeting but they declined. The full email can be found in the elicitation folder.

\subsubsection{Prototype Feedback}
The fourth elicitation was the prototype feedback where we asked for feedback on our prototype from a potential users, also the same users from the second elicitation. 
... to-do ... we have only prepared for the tests.