\subsection{Interviews of Travellers}
In order to determine traveller goals and needs as well as to validate the product idea of Easytrip, data has been elicited from potential travellers. The data was collected from semistructured interviews with 2 participants. The participants were chosen based on people that we know are frequent travelers. The interviews were conducted in a casual setting, and the recorded data is anonymous.

An interview guide was used to ensure consistency among the interviews. The guide was structured in three parts: Introduction, experience with existing travel planning tools, questions regarding the process to finding cheap flight, as well as closing questions. 

\subsubsection{Results}
The participants (N=2) had an average age of 24 years and both had simular prior experience with using travel planning tools. They both used airplane travel one or two times each ear, mostly for vacation. For shorter destinations, travel by car or train was desired to minimize their enviornmental footprint. Both believe that the most important factor when booking a flight is price, but other factors such as layovers and travel dates were also mentioned. Additionaly, they value flexibility and personalization in the booking process. 

\subsubsection{Problems}
One of the participants found it stressful how algorithms increase prices when refreshing the page. Booking through third-party platforms was also brought up as a problem since the particapnt had experienced multiple crashes. The other participant expressed frustration with the lack of transparency in pricing, with some sites having hidden baggage fees and unclear total costs making price comparisons between flights difficult. 

\subsubsection{Takeaways}


\subsubsection{Finding Cheap Flights}
Participants used different methods to find cheaper flights. One preferred sorting by the lowest available prices and using price prediction tools like those in Hopper. The other valued flexible date search options and price calendars that displayed price variation over time. This suggests a need for a tool that combines multiple features to find the best deals hinting for a flexible design depending on user preferences.


\subsection{Prototype feedback}
Purpose: Test and validate an inital design of the system, to gather feedback on the traveller interface and functionality. First impressions and usabilty of the products prototyped main features.
Scenario: The participants have a low budget and really feel an urge to travel. They are looking for a cheap flight to a sunny destination in Europe. They have a flexible schedule and are open to suggestions. They are looking for a platform that can provide them with the best deals and recommendations based on their preferences.
Data Collection: Think-out-load test to discoverQualitative questions after the test.
Participants: ?`?
TO-DO: these tests will be conducted in the future, and the results will be presented in the next release.

\subsection{Stakeholder Analysis}
Our stakeholders include competitors such as Momondo and Flight Scanner, travel agencies providing data partnerships, airline companies indirectly benefiting from bookings, end travellers, travelers, providing feedback, IT maintenance and development teams, and non-obvious actors such as the support department. Each stakeholder have different needs and goals, and it is important to understand these in order to create a successful product as well as to manage the relationships and expectations of the stakeholders.

\subsubsection{Airlines}
Since airlines are EasyTrips main source of data, information is needed regarding API endpoints for fetching price, date and times, availability based on travellers' search queries. Also, agreements on business models for affiliate links will be needed to ensure revenue channels. Additionally, real-time updates on flight status and any changes in schedules will be required to maintain accurate and up-to-date information for our travellers.


Airlines benefit from increased visibility, effectively using EasyTrip as a marketing and sales channel. By providing accurate and real-time data, airlines can attract more customers and increase their revenue through the platform. The competition between airlines is also reflected in the platform, as travellers will compare prices and services, increasing the pressure on airlines to provide competitive offers and maintain high service quality.

   
The platform has a strong dependency on airlines for providing accurate and timely data. Inconsistent or outdated flight information could negatively impact traveller experience and trust in our platform. Changes in airlines' API or business models could lead to disruptions and additional development costs and maintainment costs on our end. Airlines could withdraw from agreements, affecting our affiliate revenue stream. Any inaccuracies or delays in data could reflect poorly on their services and reputation.


Airlines may suggest improvements such as dynamic pricing or marketing campaigns to maximize mutual benefits. Potential collaboration could be prioritizing a airline in the search results, or offering exclusive deals to travellers. These collaborations could be beneficial for both parties, as it would increase the visibility of the airline and create a new revenue stream for us.

\subsubsection{Travel Agencies}
Travel agencies provide data partnerships and EasyTrip needs access to their data regarding flight offers in a similar way that we are dependent on the airlines' data.


Travel agencies benefit by reaching more customers through the platform in a similar way as for airlines. The main difference is the advantage of being able to combine multiple airline in routes including multiple layoveers. This is a service that airlines do not provide. Therefor, travel agencies can promote deals and lead customers to additional purchases since they also provide accomodations and other travel services. Both airline and travel agencies benefit from increased visibility and bookings through the platform, effectively using us as a marketing and sales channel.


There are risks that travel agencies could provide inconsistent data, leading to inaccurate package pricing or availability issues. Competition between agencies could result in disputes over featured listings. Also, there is a risk of agencies withdrawing from agreements, affecting our affiliate revenue stream. The platform is dependent on the travel agencies for providing accurate and timely data. Similar to the airline can changes to API or business models could lead to disruptions and additional development costs and maintainment costs on our end. Favoring of competitors could also be a risk, as the agencies might prioritize other platforms over EasyTrip. The agencies might have exclusive deals and partnerships with other platforms, which could limit the provided data.


Similar to the airlines is the potential to include promotional campaigns targeting high-demand destinations. This could be beneficial for both parties. However, such aggreement would need to be carefully negotiated to ensure that both parties benefit from the collaboration, without hurting the relationship to other stakeholders. For the development process, travel agencies could also provide feedback on the platform and suggest improvements to the traveller interface or features to ensure that their data is presented in the best possible way.

\subsubsection{Travellers}
To provide a personalized experience, the platform needs information about travellers budget constraints, destinations and travel dates. Optionally, personal data such as email can be gathered to enable notifications on price changes. Also, feedback on their traveller experience crucial for the improvement and future of EasyTrip.


Travellers benefit from accurate, real-time price comparisons, and personalized recommendations. They gain time efficiency and cost savings when planning trips. Additionally, travellers are the main source of revenue for the platform, as they generate bookings and affiliate revenue. Therefore, user satisfaction and loyalty are crucial for the success of the platform. By providing feedback and suggestions, travellers can influence the platform's development and improve their overall experience. This could lead to increased retention and engagement, meaning travellers will return to the platform for future travel planning.


The dependency on airlines is a factor that can generate issues such as inaccurate flight information, delayed notifications, or privacy concerns. If travellers feel their data is not secure or their preferences are not being met, they may lose trust in the platform and seek alternatives.

To mitigate these risks, EasyTrip will implement validation and consistency checks to ensure accurate flight information. Additionally, the platform will provide clear and transparent privacy policies to address any privacy concerns. Regular traveller feedback will also be gathered by interviews and forums to identify possible problems with the system. 

\subsubsection{IT Support and Maintenance}
The IT support team requires a comprehensive administration interface to monitor system errors, API updates, and data quality from airlines and travel agencies. They need access to logs, error reports, and real-time data feeds to ensure the system's integrity and performance. Since payment and booking are handled by airlines or travel agencies, the IT support team does not need access to payment information, allowing them to focus on technical issues and system maintenance.


The IT support team is crucial for maintaining the platform's reliability and performance. They ensure that data from airlines and travel agencies is accurate and up-to-date, which is essential for traveller trust and satisfaction. By resolving technical issues promptly and ensuring seamless API integrations, the IT support team helps maintain the platform's reputation and operational efficiency.


The IT support team faces risks such as system downtime, and technical issues that could disrupt the platform's functionality. Not enough monitoring and delayed response to issues can lead to prolonged outages, negatively impacting traveller experience and trust. Additionally, the team must stay updated with API changes from airlines and travel agencies to prevent integration issues. To resolve this, a error handling system must be in place to detect and address issues promptly.


To mitigate these risks, the IT support team will implement robust monitoring and alerting systems to detect and address issues promptly. Regular updates and maintenance schedules will be established to ensure system stability. Collaboration with airlines and travel agencies will be maintained to stay informed about API changes and updates. Additionally, the team will conduct regular data quality checks and validation processes to ensure the accuracy and reliability of the information presented on the platform.

\subsubsection{Competitors - Momondo, Flight Scanner}
EasyTrip needs detailed information about the features, pricing models, and user interfaces of competitors like Momondo and Flight Scanner. Specifically, insights into how they handle flight data aggregation, display recommendations, and optimize search algorithms are critical. Also, understanding their their business model will be important since EasyTrip will have similar revenue streams as the competitors, using affiliate marketing strategies and partnerships with airlines.


Competitors may benefit from increased visibility and bookings, similar to EasyTrip. However, their primary stake is competitive, as they aim to attract and retain a larger user base. Their ability to secure exclusive partnerships may impact our access to data and revenue streams. However, since EasyTrip is rather for finder wherer the cheapest flight is, the competition has a different perspective on the market. They are more focused on providing a flight price comparisons for travelers already knowing their destination.


Competitors have several risks to EasyTrip, such as attracting potential travellers with more competitive pricing, better deals, or superior user experience. Established competitors can also secure exclusive data partnerships, limiting EasyTrip’s access to certain deals or flight information. Also, their already established brand loyalty can make it challenging for EasyTrip to gain market shares.



To minimize the risks, EasyTrip can focus on differentiating itself through innovative features, such as visualization of flight data. EasyTrip can also emphasize user experience by providing intuitive search tools. Collaboration with smaller travel agencies or airlines could help secure unique offers that competitors may not have access to. EasyTrip can also leverage feedback from users to continuously improve its platform and stay competitive.


\subsection{Conflicting Stakeholder Interests}
The table below explains the key conflicts between stakeholders and possible solutions for resolving them:

\begin{longtable}{|p{4cm}|p{6cm}|p{6cm}|}
    \hline
    \rowcolor{headergray}
    \textbf{Stakeholders Involved} & \textbf{Conflicting Interests} & \textbf{Potential Solutions} \\ \hline
    Airlines vs. Travel Agencies & Airlines prefer direct bookings, while travel agencies benefit from combining routes & Negotiate fair partnerships where travel agencies promote airline offers and bundle other services effectively. \\ \hline
    Airlines vs. EasyTrip & Airlines may want preferential treatment in search results & Maintain neutrality by implementing a transparent ranking system based on user preferences, not payments. \\ \hline
    Competitors vs. EasyTrip & Competing for user base and exclusive airline partnerships & Differentiate through unique features, personalized services, and feedback-driven innovation. \\ \hline
    Travellers vs. Airlines & Travellers want low prices, while airlines focus on maximizing profit & Implement dynamic pricing with filters to show budget-friendly options alongside premium offers. \\ \hline
    IT Support vs. Airlines/Agencies & API changes by airlines or agencies may cause disruptions & Maintain regular communication with API providers and implement fallback mechanisms to reduce downtime. \\ \hline
\end{longtable}

\subsubsection{Resolving Conflicts}
To resolve conflicts, EasyTrip will establish internal clear and transparent guidelines for collaboration and prioritize creating beneficial agreements for multiple stakeholders. Keeping strong relationships using regular communication with airlines and travel agencies is important, especially due to our dependencies on their data. Feedback from travellers will probably be the greatest prioritization, ensuring user satisfaction without compromising data partnerships. A robust monitoring system will be implemented to manage technical dependencies and reduce the risk of conflicts affecting platform performance.

\subsubsection{Alternative Data Sources - Flight Scanner API}
In case of losing access to airline data, EasyTrip will use Flight Scanner API as an alternative data source. Flight Scanner provides similar data on flight prices, availability, and schedules, which could be used as a backup plan. However, this would require a different business model, as Flight Scanner does not provide affiliate marketing strategies. Instead, EasyTrip could rely solely on ads as a revenue stream. This alternative data source could be used as a backup plan in case of disruptions, changes in airline data access and for distribution of risk.
\subsection{Elicitation of easyJet}
To futher enhance our stakeholder analysis and gather more information about the stakeholders, both competitors such as Momondo and Flight Scanner, as well as airlines such as SAS and Norwegian and 45 other airlines, were contacted. The purpose of the contact was to gather information about their needs, pricing strategies, and API endpoints. 

EasyJet is a low-cost airline that operates in Europe. The airline was contacted via email and questions were asks about traveler challanges, API endpoints and limitations, data display requirements and other considerations. The following information was gathered from the contact:
\subsubsection{Challenges Faced by Travellers}
EasyJet highlighted that while its booking system is designed to be user-friendly, travelers often face difficulties when trying to manage bookings across different platforms such as third party solutions like Travel Agencies. Seamless integration of third-party platforms with airline booking systems is a key challenge that can be improved upon to enhance the user experience.

\subsubsection{API Integration and Real-Time Data}
EasyJet pointed out the importance of API integration that supports real-time updates on flight schedules, prices, and availability. While they did not reveal specific technical details about their API agreements, they recommended that any travel planning tool prioritize accuracy, data privacy, and security. This ensures users have up-to-date information and maintain control over their personal data.

\subsubsection{Display Requirements}
Although easyJet did not provide detailed requirements for how their flights and promotions should be displayed on third-party websites, they underscored the importance of maintaining high data standards and compliance. Platforms like EasyTrip would need to adhere to data usage agreements and provide transparent privacy policies.

\subsubsection{Use of Publicly Available Resources}
Due to prioritization on enhancing user experiences on their own platforms, easyJet was unable to offer a dedicated meeting. Also they were unable to share details about their pricing strategies or data-sharing agreements with our competitors. However, they encouraged us to use publicly available resources on their website for information regarding booking processes, and user services.

\subsubsection{Underutilized Features in Flight Comparison Tools}
Although easyJet did not specify which features they believe are underutilized in current flight comparison systems, they truly emphasized on data accuracy, privacy, and ease of booking management. This suggests that addressing these areas could lead to significant improvements.

\subsubsection{Takeaways}
\begin{enumerate}
    \item EasyTrip will need to explore API partnerships that prioritize real-time data updates, potentially collaborating with services like Flight Scanner as a backup plan or as the primary data source in the minimal viable product.
    \item Enhancing seamless management of bookings across platforms could address key pain points identified by easyJet. Excluding booking features could be a strategic decision to avoid direct competition with airlines and to focus on providing a user-friendly flight comparison tool. This resolve the conflict of interest between airlines and EasyTrip.
    \item Implementing clear data privacy policies and ensuring compliance with third-party agreements will be critical to avoid any data-related issues. This will help build trust with airlines and travelers, and maintain a positive reputation in the industry. The use of cookies and data collection should be transparent and in accordance with GDPR regulations to favor the travellers.
\end{enumerate}