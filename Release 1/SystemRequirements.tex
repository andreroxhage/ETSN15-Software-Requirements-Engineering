\documentclass[a4paper]{article}
\usepackage[T1]{fontenc}
\usepackage[english]{babel}
\usepackage[utf8]{inputenc}
\usepackage[margin=3cm]{geometry}
% \usepackage{amsmath}
% \usepackage{amssymb}
\usepackage{parskip}
% \usepackage{csquotes}
% \usepackage[hyperref,backend=biber,style=ieee,isbn=true,urldate=iso8601]{biblatex}
\usepackage[colorlinks=true, allcolors=blue]{hyperref}
% \usepackage{listings}
\usepackage{graphicx}

%% Hyperref options
\makeatletter
\hypersetup{
    linktoc=all,
    unicode=true,
    pdfcreator=,
    pdftitle={\@title},
    pdfauthor={\@author},
    pdfdisplaydoctitle=true,
}

\title{System Requirements}
\author{Jonathan Ahlström \and Ossian Gewert \and Jacob Jönsson \and Simon Persson \and André Roxhage \and Felix Sundholm}

\begin{document}
\maketitle

\begin{center}
    ETSN15 Requirements Engineering

Group Gamma - EasyTrip

% Jonathan Ahlström
% Ossian Gewert
% Jacob Jönsson
% Simon Persson
% André Roxhage
% Felix Sundholm
\end{center}

\tableofcontents

\section{Introduction}
System Requirements include the following:
% System Requirements include the following:
% (a) Different types of system requirements, e.g. data, function, quality
% at different levels, e.g. goal, domain, product, design.
% (b) Several specification techniques, e.g. context diagrams, features,
% virtual windows, task descriptions.
% (c) Each requirement should have a unique identity (name or number). FQ1, Q1, D1,
% (d) A Release Plan should be part of R2 & R3, in which a subset of the
% requirements are prioritized and planned to be implemented in three
% imagined future product releases called R4, R5, R6.
% (e) Design-level requirements should be specified for the sub-set of requirements that are planned for a future imagined product release R4 - Jonathan & André
% (see 6e). This sub-set of requirements shall be implemented as mockup designs in R3 using, e.g. screens and prototypes, analog drawings, clickable presentations, executable GUI mockups or similar

\section{Elicitation}
% Jonathan & André
% Describe the elicitation process, what went well, what could have been done better. Interview, 1-2 persons, inteview guide, semistructural interview. Then summarize the results in the elixitaion.tex.
\subsection{Elicitation of potential travellers}
In order to determine traveller goals and needs as well as to validate the product idea of Easytrip, we have elicited data from potential travellers. The data was collected from semistructured interviews with 2 participants. The participants were chosen based on people that we know are frequent travelers. The interviews were conducted in a casual setting, and the recorded data is anonymous.

A interview guide was used to ensure consistency among the interviews. The guide was structured in three parts: Introduction, experience with existing travel planning tools, questions regarding the process to finding cheap flight, as well as closing questions. 

\subsubsection{Results}
The participants (N=2) had an average age of 24 years and both had experience with using travel planning tools. The participants had different preferences when it came to finding cheap flights. 
... to-do sammanfatta resultaten, sub sub sub section?💕 ...

\subsection{Prototype feedback}
Purpose: Test and validate an inital design of the system, to gather feedback on the traveller interface and functionality.

\subsection{Stakeholder Analysis}
Our stakeholders include competitors such as Momondo and Flight Scanner, travel agencies providing data partnerships, airline companies indirectly benefiting from bookings, end travellers, travelers, providing feedback, IT maintenance and development teams, and non-obvious actors such as the support department. Each stakeholder have different needs and goals, and it is important to understand these in order to create a successful product as well as to manage the relationships and expectations of the stakeholders.

\subsubsection{Airlines}
\subsubsubsection{Information Needed}
Since airlines are EasyTrips main source of data, information is needed regarding API endpoints for fetching price, date and times, availability based on travellers' search queries. Also, agreements on business models for affiliate links will be needed to ensure revenue channels. Additionally, real-time updates on flight status and any changes in schedules will be required to maintain accurate and up-to-date information for our travellers.

\subsubsubsection{Stake in the Platform}
Airlines benefit from increased visibility, effectively using EasyTrip as a marketing and sales channel. By providing accurate and real-time data, airlines can attract more customers and increase their revenue through the platform. The competition between airlines is also reflected in the platform, as travellers will compare prices and services, increasing the pressure on airlines to provide competitive offers and maintain high service quality.

\subsubsubsection{Risks}   
The platform has a strong dependency on airlines for providing accurate and timely data. Inconsistent or outdated flight information could negatively impact traveller experience and trust in our platform. Changes in airlines' API or business models could lead to disruptions and additional development costs and maintainment costs on our end. Airlines could withdraw from agreements, affecting our affiliate revenue stream. Any inaccuracies or delays in data could reflect poorly on their services and reputation.

\subsubsubsection{Solutions and Resources}
Airlines may suggest improvements such as dynamic pricing or marketing campaigns to maximize mutual benefits. Potential collaboration could be prioritizing a airline in the search results, or offering exclusive deals to travellers. These collaborations could be beneficial for both parties, as it would increase the visibility of the airline and create a new revenue stream for us.

\subsubsection{Travel Agencies}
\subsubsubsection{Information Needed}
Travel agencies provide data partnerships and EasyTrip needs access to their data regarding flight offers in a similar way that we are dependent on the airlines' data.

\subsubsubsection{Stake in the Platform}
Travel agencies benefit by reaching more customers through the platform in a similar way as for airlines. The main difference is the advantage of being able to combine multiple airline in routes including multiple layoveers. This is a service that airlines do not provide. Therefor, travel agencies can promote deals and lead customers to additional purchases since they also provide accomodations and other travel services. Both airline and travel agencies benefit from increased visibility and bookings through the platform, effectively using us as a marketing and sales channel.

\subsubsubsection{Risks}
There are risks that travel agencies could provide inconsistent data, leading to inaccurate package pricing or availability issues. Competition between agencies could result in disputes over featured listings. Also, there is a risk of agencies withdrawing from agreements, affecting our affiliate revenue stream. The platform is dependent on the travel agencies for providing accurate and timely data. Similar to the airline can changes to API or business models could lead to disruptions and additional development costs and maintainment costs on our end. Favoring of competitors could also be a risk, as the agencies might prioritize other platforms over EasyTrip. The agencies might have exclusive deals and partnerships with other platforms, which could limit the provided data.

\subsubsubsection{Solutions and Resources}
Similar to the airlines is the potential to include promotional campaigns targeting high-demand destinations. This could be beneficial for both parties. However, such aggreement would need to be carefully negotiated to ensure that both parties benefit from the collaboration, without hurting the relationship to other stakeholders. For the development process, travel agencies could also provide feedback on the platform and suggest improvements to the traveller interface or features to ensure that their data is presented in the best possible way.

\subsubsection{Travellers}
\subsubsubsection{Information Needed}
To provide a personalized experience, the platform needs information about travellers budget constraints, destinations and travel dates. Optionally, personal data such as email can be gathered to enable notifications on price changes. Also, feedback on their traveller experience crucial for the improvement and future of EasyTrip.

\subsubsubsection{Stake in the Platform}
Travellers benefit from accurate, real-time price comparisons, and personalized recommendations. They gain time efficiency and cost savings when planning trips. Additionally, travellers are the main source of revenue for the platform, as they generate bookings and affiliate revenue. Therefore, user satisfaction and loyalty are crucial for the success of the platform. By providing feedback and suggestions, travellers can influence the platform's development and improve their overall experience. This could lead to increased retention and engagement, meaning travellers will return to the platform for future travel planning.

\subsubsubsection{Risks}
The dependency on airlines is a factor that can generate issues such as inaccurate flight information, delayed notifications, or privacy concerns. If travellers feel their data is not secure or their preferences are not being met, they may lose trust in the platform and seek alternatives.

\subsubsubsection{Solution och Resources}
To mitigate these risks, EasyTrip will implement validation and consistency checks to ensure accurate flight information. Additionally, the platform will provide clear and transparent privacy policies to address any privacy concerns. Regular traveller feedback will also be gathered by interviews and forums to identify possible problems with the system. 

\subsubsection{IT Support and Maintenance}
\subsubsubsection{Information Needed}
The IT support team requires a comprehensive administration interface to monitor system errors, API updates, and data quality from airlines and travel agencies. They need access to logs, error reports, and real-time data feeds to ensure the system's integrity and performance. Since payment and booking are handled by airlines or travel agencies, the IT support team does not need access to payment information, allowing them to focus on technical issues and system maintenance.

\subsubsubsection{Stake in the Platform}
The IT support team is crucial for maintaining the platform's reliability and performance. They ensure that data from airlines and travel agencies is accurate and up-to-date, which is essential for traveller trust and satisfaction. By resolving technical issues promptly and ensuring seamless API integrations, the IT support team helps maintain the platform's reputation and operational efficiency.

\subsubsubsection{Risks}
The IT support team faces risks such as system downtime, and technical issues that could disrupt the platform's functionality. Not enough monitoring and delayed response to issues can lead to prolonged outages, negatively impacting traveller experience and trust. Additionally, the team must stay updated with API changes from airlines and travel agencies to prevent integration issues. To resolve this, a error handling system must be in place to detect and address issues promptly.

\subsubsubsection{Solutions and Resources}
To mitigate these risks, the IT support team will implement robust monitoring and alerting systems to detect and address issues promptly. Regular updates and maintenance schedules will be established to ensure system stability. Collaboration with airlines and travel agencies will be maintained to stay informed about API changes and updates. Additionally, the team will conduct regular data quality checks and validation processes to ensure the accuracy and reliability of the information presented on the platform.

\subsubsection{Competitors - Momondo, Flight Scanner - TOOO DOOO 💩}
\subsubsubsection{Information Needed}
Write this!!! Using the bullet poitnts isch 🤓
\begin{itemize}
    \item \textbf{User Experience Feedback:} Gather reviews and feedback from users of competitor platforms to identify common pain points and areas for improvement.
    \item \textbf{Feature Comparison:} Compare the features offered by competitors, such as search filters, booking options, and user interface design, to identify gaps and opportunities for innovation.
    \item \textbf{Pricing Models:} Understand the pricing strategies and commission structures of competitors to ensure competitive pricing and better deals for travellers.
    \item \textbf{Market Positioning:} Analyze how competitors position themselves in the market, including their marketing strategies and target demographics, to identify potential niches or underserved segments.
    \item \textbf{Technical Performance:} Evaluate the technical performance of competitor platforms, including website speed, mobile app functionality, and API reliability, to ensure EasyTrip offers a superior technical experience.
    \item \textbf{Customer Support:} Assess the quality and availability of customer support provided by competitors to identify areas where EasyTrip can offer better service.
\end{itemize}

\subsubsubsection{Stake in the Platform}
Simularly to EasyTrip, competitors benefit from increased visibility and bookings through their platform. 

\subsubsubsection{Risks}
Competitors are heavily reliant on the flight industry, as their services would not generate any income if there were no active flights. Their business model and traveller base are directly tied to the demand for air travel.


\subsubsubsection{Solutions and Resources}
Competitors have an established role in the industry, providing travellers with alternative platforms for flight comparisons and bookings. 

\subsection{Conflicting Stakeholder Interests}
Make a table with conflicting interests and how to solve them

Write a paragrapgh about how to solve them and the stakeholders relationship to each other

Also if we dont get access or lose acces a backup plan would be to use FLight scanners api and then solely use ads a revenue stream
https://developers.skyscanner.net/docs/intro

\subsection{Contacted Stakeholders - Competitors & Airlines}
To futher enhance our stakeholder analysis and gather more information about the stakeholders, both competitors such as Momondo and Flight Scanner, as well as airlines such as SAS and Norwegian and 45 other airlines, were contacted. The purpose of the contact was to gather information about their needs, pricing strategies, and API endpoints. 
... finish writing this sentence ??? did they replay to us or not etc ... ???

\section{Functional Requirements}
% Ossian
% Describe what the system does, i.e., the services it provides to the users.
\subsection{Functional Requirements}
Introduction



\subsection{Data dictionary}
add a explanaiton and a table of the data dictionary


\section{Data / Non-Functional Requirements}
% Jacob & Felix
% Details about the data the system will handle, API's etc, data structure and aggregations
\subsection{Data Requirements}

\subsubsection{Data Model}
The design level data requirements are represented through a E/R diagram:

\includegraphics{resources/dataRelations.PNG}

\subsubsection{Data Dictionary}

DR2: Domain level data requirement

Class: User
The user is a Traveller or Travel Agency who has a user account in the product.

Examples:
1. A traveller who has a user account.
2. A travel agency who uses their user account to search flights.

Attributes:
email:      Text, 320 chars
            The user's email address. This email address is used for communication with the user outside the product.

name:       Text, 50 chars
            The name of the traveller or travel agency using the account.

newsletter: Boolean
            Wether the user wants mass email's from the product or not.

city:       Text, 35 chars
            The city the user's default origin is set to. This is the standard origin used when performing searches for the user.

\subsubsection{Virtual window}
Create a virtual window that shows the data that is to be displayed in the system. 

See this example 

\begin{longtable}{|p{3cm}|p{9cm}|}
    \hline
    \rowcolor{headergray}
    \textbf{Field} & \textbf{Value} \\ \hline
    \textbf{Email} & exampleJohn@gmail.com \\ \hline
    \textbf{Name} & John Doe \\ \hline
    \textbf{Utskick} & True \hspace{1em} False \\ \hline
    \textbf{City} & New York \\ \hline
\end{longtable}

\vspace{1cm} % Space between tables

\begin{longtable}{|p{3cm}|p{3cm}|p{3cm}|}
    \hline
    \rowcolor{headergray}
    \textbf{From} & \textbf{Price Max} & \textbf{To} \\ \hline
    NY & 3000 SEK & CPH \\ \hline
    CPH & 2500 SEK & NY \\ \hline
    ... & ... & ... \\ \hline
\end{longtable}

\section{Quality Requirements}
% Simon Persson
% Describe the quality attributes of the system, such as performance, usability, etc.
\subsection{Quality Requirements}
intro etc



\section{Context Diagram}
% André can update this if needed
% Explanation of diagram and its components
\includegraphics[width=.99\textwidth]{resources/contextDiagram.png}

\section{Mockups}
% André & Jonathan
% Figma wireframes or lofi prototypes
\includegraphics[width=.99\textwidth]{mockups/mockup1.png}

\end{document}